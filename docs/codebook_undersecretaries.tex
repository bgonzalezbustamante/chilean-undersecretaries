%%%%%%%%%%%%%%%%%%%%%%%%%%%%%%%%%%%%%%%%%%%%%%%%%%

%!TEX TS-program = xelatex
%!TEX encoding = UTF-8 Unicode

%%%%%%%%%%%%%%%%%%%%%%%%%%%%%%%%%%%%%%%%%%%%%%%%%%

%% Generated by the codebookr R package
%% Joshua C. Fjelstul

%% Modified by Bastián González-Bustamante
%% University of Oxford
%% https://github.com/bgonzalezbustamante/chilean-ministers

%% Creative Commons Attribution 4.0 International License
%% https://github.com/bgonzalezbustamante/chilean-ministers/blob/master/LICENSE.md

%%%%%%%%%%%%%%%%%%%%%%%%%%%%%%%%%%%%%%%%%%%%%%%%%%

%% Packages and Fonts

%%%%%%%%%%%%%%%%%%%%%%%%%%%%%%%%%%%%%%%%%%%%%%%%%%

\documentclass[10pt]{article}

%% Packages
\usepackage{geometry}
\usepackage{graphicx} 
\usepackage{xcolor}
\usepackage{tikz} 
\usepackage{setspace}
\usepackage{url}
\usepackage{ragged2e}
\usepackage{longtable}
\usepackage{enumitem}
\usepackage{atbegshi} 
\usepackage{tcolorbox}

%% Fonts
\usepackage[english]{babel}
\usepackage{underscore}
\usepackage{anyfontsize}
\usepackage[utf8]{inputenc}
\usepackage[T1]{fontenc}
\usepackage{fontspec}

%% ToC
\usepackage{tocloft}

%% Colors
\definecolor{themecolor}{HTML}{002147}
\definecolor{background}{HTML}{EEF6FD}

%% Hyperlinks
\usepackage[colorlinks=true, linkcolor=themecolor, citecolor=themecolor, urlcolor=themecolor, breaklinks=true]{hyperref}

%%%%%%%%%%%%%%%%%%%%%%%%%%%%%%%%%%%%%%%%%%%%%%%%%%

%% Format

%%%%%%%%%%%%%%%%%%%%%%%%%%%%%%%%%%%%%%%%%%%%%%%%%%

%% Main Font
\setmainfont[Ligatures=TeX,BoldFont={Roboto Medium}]{Roboto Light}
\setmonofont[Ligatures=TeX]{Roboto Mono-Light}

%% Margins and Size
\geometry{top = 1.5in, bottom = 1.5in, left = 1.5in, right = 1.5in}
\geometry{letterpaper}

%% Format ToC
\renewcommand{\cftsecdotsep}{10}
\renewcommand{\cftsecleader}{\cftdotfill{\cftdotsep}}
\renewcommand{\cftsecfont}{{\small\color{black!75}\bfseries}}
\renewcommand{\cftsecpagefont}{{\small\color{black!75}\normalfont}}

%% Spacing
\usepackage{parskip}
\parskip=10pt
\renewcommand{\baselinestretch}{1.4}

%% Hyphen
\hyphenpenalty = 10000
\exhyphenpenalty = 10000

%% Orphan Lines
\widowpenalty10000
\clubpenalty10000

%%%%%%%%%%%%%%%%%%%%%%%%%%%%%%%%%%%%%%%%%%%%%%%%%%

%% Page Elements

%%%%%%%%%%%%%%%%%%%%%%%%%%%%%%%%%%%%%%%%%%%%%%%%%%

%% Code Box
\newtcbox{\codebox}{nobeforeafter,tcbox raise base,colback=black!5,colframe=white,coltext=black!75,boxrule=0pt,arc=3pt,boxsep=0pt,
left=4pt,right=4pt,top=3pt,bottom=3pt}

%% Chip
\newtcbox{\chip}{nobeforeafter,tcbox raise base,colback=black!5,colframe=white,coltext=black!75,boxrule=0pt,arc=11pt,boxsep=0pt,
left=10pt,right=10pt,top=8pt,bottom=8pt}

%% Format Code
\newcommand{\code}[1]{\codebox{{\footnotesize\texttt{#1}}}}

% Highlight Text
\newcommand{\highlight}[1]{{\color{themecolor} \textbf{#1}}}

%% Divider
\newcommand{\dividerline}{{\color{gray!10} \rule[4pt] {\textwidth}{3pt}}}

%% Cover
\newcommand{\cover}[4]{
\begin{tikzpicture}[remember picture,overlay, shift={(current page.south west)}]
\fill[themecolor] (0, 5.5in) rectangle ++ (8.5in, 5.5in); % header bar
\fill[black!5] (0, 4in) rectangle ++ (8.5in, 1.5in); % middle bar
\fill[white] (0, 0in) rectangle ++ (8.5in, 4in); % footer bar
\node[anchor=west] at (1.5in, 6.25in) {\color{white} \fontsize{60}{60}\selectfont \begin{minipage}{5.5in} \textbf{Codebook} \fontsize{15}{15}\selectfont \hspace{5pt} v #2 \end{minipage}};
\node[anchor=west, align=left] at (1.5in, 4.75in) {\begin{minipage}{5.5in} \color{black!40} \fontsize{#4}{#4} \selectfont #1 \end{minipage}};
\node[anchor=west, align=left, minimum height=2in] at (1.5in, 2.55in) {\begin{minipage}[t][2in]{5.5in} \color{black!40} \fontsize{10}{10} \selectfont #3 \end{minipage}};
\end{tikzpicture}
}

%% Header 
\newcommand{\headerpage}[4]{
	\newpage
	\begin{tikzpicture}[remember picture,overlay, shift={(current page.south west)}]
		\fill[themecolor] (0, 9in) rectangle ++ (8.5in, 2in); % header line 1
		\fill[black!5] (0, 8in) rectangle ++ (8.5in, 1in); % header line 2
		\node[anchor = west] at (1.5in, 9.6in) {\color{white} \fontsize{#3}{#3}\selectfont \textbf{#1}}; % heading
		\node[anchor = west] at (1.5in, 8.5in) {\color{black!40} \fontsize{#4}{#4}\selectfont #2}; % heading
	\end{tikzpicture}
	\phantomsection
	\addcontentsline{toc}{section}{#1}
	\vspace{1.5in}
}

%% Layout Page
\newcommand\pagelayout{
	\begin{tikzpicture}[remember picture,overlay, shift={(current page.south west)}]
		% \fill[themecolor] (0, 10.75in) rectangle ++ (8.5in, 0.25in); % header
		\fill[black!5] (0, 0) rectangle ++ (8.5in, 0.5in); % footer
		\draw (0.25in, 0.25in) node[anchor = west] {\fontsize{9}{9}\selectfont \color{black!40} Data Set on Chilean Undersecretaries (1990-2022) \hspace{5pt} | \hspace{5pt} González-Bustamante and Olivares (2022)}; % footer content
		\draw (8.25in, 0.25in) node[anchor = east] {\fontsize{9}{9}\selectfont \color{black!40} \thepage}; % page number
	\end{tikzpicture}
}

\AtBeginShipout{
	\AtBeginShipoutUpperLeft{\pagelayout}
}

%% Subheading
\newcommand{\subheading}[1]{
\vspace{24pt}
{\color{themecolor} \fontsize{14}{14}\selectfont \textbf{#1}}
\vspace{6pt}
\dividerline
\vspace{-20pt}
}

%%%%%%%%%%%%%%%%%%%%%%%%%%%%%%%%%%%%%%%%%%%%%%%%%%

%% Document

%%%%%%%%%%%%%%%%%%%%%%%%%%%%%%%%%%%%%%%%%%%%%%%%%%

\begin{document}

\clearpage
\pagestyle{empty}

\color{black!75}

\small

\begin{flushleft}

%%%%%%%%%%%%%%%%%%%%%%%%%%%%%%%%%%%%%%%%%%%%%%%%%%

%% Cover

%%%%%%%%%%%%%%%%%%%%%%%%%%%%%%%%%%%%%%%%%%%%%%%%%%

\cover{Data Set on Chilean Undersecretaries (1990-2022) \\ \href{https://doi.org/10.5281/zenodo.5715384}{\small https://doi.org/10.5281/zenodo.5715384}}{2.0.0}{Bastián González-Bustamante \\ University of Oxford \\ \href{https://doi.org/10.5281/zenodo.5715384}{\footnotesize https://orcid.org/0000-0003-1510-6820} \vspace{3mm} \\ Alejandro Olivares \\ Universidad Católica de Temuco, Chile \\\href{https://orcid.org/0000-0001-6934-2437}{\footnotesize https://orcid.org/0000-0001-6934-2437}}{16}

\newpage

%%%%%%%%%%%%%%%%%%%%%%%%%%%%%%%%%%%%%%%%%%%%%%%%%%

%% ToC

%%%%%%%%%%%%%%%%%%%%%%%%%%%%%%%%%%%%%%%%%%%%%%%%%%

%% Reset Page Counter
\setcounter{page}{1}

%% Format ToC and  Header
% \renewcommand\contentsname{{\color{themecolor} \fontsize{14}{14}\selectfont Datasets}}
\renewcommand\contentsname{\subheading{Table of Contents} \vspace{0pt}}

%% ToC
\tableofcontents
\addtocontents{toc}{\protect\thispagestyle{empty}}
\newpage

%%%%%%%%%%%%%%%%%%%%%%%%%%%%%%%%%%%%%%%%%%%%%%%%%%

%% Getting Started

%%%%%%%%%%%%%%%%%%%%%%%%%%%%%%%%%%%%%%%%%%%%%%%%%%

\headerpage{Getting Started}{Import Data}{30}{12}

\subheading{R Code}

\begin{verbatim}
github_1 <- "https://raw.githubusercontent.com/"
github_2 <- "bgonzalezbustamante/chilean-undersecretaries/main/data/tidy/"

chilean_ministers <- read.csv(paste(github_1, github_2, 
                              "chl_undersecretaries.csv", sep = ""),
                               header = T, sep = ",", encoding = "UTF-8")
\end{verbatim}

\subheading{Python Code}

\begin{verbatim}
import pandas as pd

github_1 = "https://raw.githubusercontent.com/"
github_2 = "bgonzalezbustamante/chilean-undersecretaries/main/data/tidy/"

url = github_1 + github_2 + "chl_undersecretaries.csv"
df = pd.read_csv(url, index_col=0)
\end{verbatim}

%%%%%%%%%%%%%%%%%%%%%%%%%%%%%%%%%%%%%%%%%%%%%%%%%%

%% Data Set

%%%%%%%%%%%%%%%%%%%%%%%%%%%%%%%%%%%%%%%%%%%%%%%%%%

\headerpage{chl\_undersecretaries}{Case-Level Data}{30}{12}

\subheading{Overview}

This data set contains information on Chilean undersecretaries between 1990 and 2022 in Comma-Separated Values \code{CSV} format with Unicode encoding \code{UTF-8}. The data collection was carried out based on official sources such as archives of Congress and ministries, the National Library, and press archives. This set contains 424 observations.

\subheading{Variables}

\begin{description}[labelwidth=130pt, leftmargin=\dimexpr\labelwidth+\labelsep\relax, font=\normalfont, itemsep=10pt]
\item[\code{id}] \code{numeric}\hspace{5pt}Unique ID for each undersecretary-portfolio observation.
\item[\code{country}] \code{string}\hspace{5pt}Country name.
\item[\code{name}] \code{string}\hspace{5pt}Officeholder name.
\item[\code{sex}] \code{string}\hspace{5pt}Officeholder sex. 
\item[\code{president}] \code{string}\hspace{5pt}President in office. 
\item[\code{start\_president}] \code{date}\hspace{5pt}Start date of presidential term in the format \code{YYYY-MM-DD}.
\item[\code{end\_president}] \code{date}\hspace{5pt}End date of presidential term in the format \code{YYYY-MM-DD}.
\item[\code{ministry}] \code{string}\hspace{5pt}Ministry name. \\ \vspace{2mm}{\footnotesize A number of cases experienced a change of name of the ministry during their time at the office. For example, the case \code{ID-261} experienced a change of name in October 2011. Considering that the observations in this set correspond to undersecretary-portfolio cases, we could have generated a new observation when that ministry changed its name. Because of the scarcity of these cases, we have not generated new observations, however, the ministries' names reflect this situation with a slash. On the other hand, the labels \code{SEGEGOB}, \code{SEGPRES}, and \code{SERNAM} correspond to the following ministries: {\itshape Secretaría General de Gobierno}, {\itshape Secretaría General de la Presidencia}, and {\itshape Servicio Nacional de la Mujer}.}
\item[\code{undersec}] \code{string}\hspace{5pt}Portfolio name.
\item[\code{start\_undersec}] \code{date}\hspace{5pt}Officeholder start date in the format \code{YYYY-MM-DD}.
\item[\code{end\_undersec}] \code{date}\hspace{5pt}Officeholder end date in the format \code{YYYY-MM-DD}.
\item[\code{party}] \code{string}\hspace{5pt}Officeholder political party \\ \vspace{2mm}{\footnotesize The labels correspond to the following political parties: \code{EVO} ({\itshape Evolución Política}), \code{IC} ({\itshape Izquierda Cristiana}), \code{MAS} ({\itshape Movimiento Amplio Social}), \code{PCCh} ({\itshape Partido Comunista de Chile}), \code{PDC} ({\itshape Partido Demócrata Cristiano}), \code{PH} ({\itshape Partido Humanista}), \code{PPD} ({\itshape Partido por la Democracia}), \code{PR} ({\itshape Partido Radical}), \code{PRI} ({\itshape Partido Regionalista Independiente}), \code{PRSD} ({\itshape Partido Radical Socialdemócrata}), \code{PS} ({\itshape Partido Socialista de Chile}), \code{RN} ({\itshape Renovación Nacional}), and \code{UDI} ({\itshape Unión Demócrata Independiente}). Finally, \code{NP} is for non-partisan undersecretaries, which could be recoded as a \code{dummy} variable.}
\end{description}

%%%%%%%%%%%%%%%%%%%%%%%%%%%%%%%%%%%%%%%%%%%%%%%%%%

%% License

%%%%%%%%%%%%%%%%%%%%%%%%%%%%%%%%%%%%%%%%%%%%%%%%%%

\headerpage{License}{Creative Commons Attribution 4.0 International license (CC BY 4.0)}{30}{12}

This open-access license allows the data to be shared, reused, adapted as long as appropriate acknowledgement is given.

%%%%%%%%%%%%%%%%%%%%%%%%%%%%%%%%%%%%%%%%%%%%%%%%%%

%% Citation

%%%%%%%%%%%%%%%%%%%%%%%%%%%%%%%%%%%%%%%%%%%%%%%%%%

\headerpage{Citation}{If you use this data set, please cite it as below}{30}{12}

González-Bustamante, B., \& Olivares, A. (2022). Data Set on Chilean Undersecretaries (1990-2022) (Version 2.0.0 -- Sparkling Shape) [Data set]. DOI: \href{https://doi.org/10.5281/zenodo.5715384}{10.5281/zenodo.5715384}

\end{flushleft}

\end{document}
